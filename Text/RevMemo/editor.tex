\section{Editor}

The reviewers have generally recommended publication, but also suggest some minor--and in case of one reviewer (R1) major--revisions to your manuscript. Therefore, I invite you to respond to the reviewers' comments and revise your manuscript.

You will notice that there is general agreement on how the ms. could be strengthened, but that the reviewers differ markedly on how fundamental these changes are perceived to be.

First of all, the reviewers are concerned about the statistical properties of the introduced statistical (tensor) model. Reviewer 1 asks for these statistical properties to be either stated explicitly or for a reference to a ``technical'' article where these properties are derived. Reviewer 3 questions what assumptions about the error term (sigma in 3) are made. I agree that these are crucial points to address in the revisions. \\

\textcolor{blue}{\emph{
	We have more clearly added in references to the ``technical'' article that this paper applies to the study of coevolving longitudional networks in IR. There is a technical paper forthcoming in the Annals of Applied Statistics that discusses the technical properties of the model we introduce here in much greater depth. This current JPR paper is meant to provide an introduction and application of this model for scholars working in international relations. However, we have clarified a number of the methodological points raised. Specifically, we have provided a version of the multilinear tensor regression (MLTR) model for a single observation, provided more detail on the Tucker product, discussed how the MLTR framework accounts for nodal heterogeneity, and further clarifies how this framework differs from extant approaches.
}} \\

The second concern that comes up repeatedly is the comparison with alternative approaches. Reviewer 1 points towards Multilayer models (Mucha) while reviewer 3 refers to MRQAP models. In my opinion, this also relates to the main concern of the second reviewer: what is the main problem that this paper addresses? Clearly, the main audience of JPR will be mainly interested to know when tensor models are most appropriate (and when it is better to rely on alternative models). It could even be helpful to compare with ``incorrect'' models. How serious are the problems that the tensor models avoid. \\

\textcolor{blue}{\emph{
	We have added in a longer discussion comparing the existing approaches used for modeling networks with the model we illustrate here. The main problem that this paper addresses is the question of how to employ a regression based approach to study the coevolution of longitudinal networks in international relations. As we state in the ``Modeling Approach'' section, ERGM based approaches such as SAOM and TERGM have been extremely useful in studying binary, longitudinal networks while still permitting an understanding of network level dependencies. The MRQAP approach is also a valuable tool but its primary limitation is that it is a model whose goal is not to study network level dependencies but control for them through permutations in the rows and columns of the data (Krackhardt, 1988; Dekker et al., 2007). The resurgence in studying multilayer/multiplex networks in the field is a crucial development and we have added in a discussion about it to the paper. Many of the tools that have been developed in this literature are at the core of the model that we illustrate, particularly, with regards to work that has been done on the use of the Tucker Product as a tool to conduct high dimensional singular value decomposition (e.g., Dunlavy et al., 2001; Kolda \& Bader, 2009.). However, we are not aware of any method in this literature that provides a comparable regression based approach to what we illustrate here. Specifically, the regression based approach we provide allows for the analysis of coevolving networks in a longitudinal context using a tensor based representation of the vector autoregression framework. 
}} \\

Finally, Reviewer 2 specifically asks for a concrete set of examples/questions; even referring to an earlier version of the paper that was made available on-line. Re-reading the introduction I noticed several examples, but admittedly they are somewhat hidden. Restructuring the introduction and/or using a specific (but preferably published) example could strengthen the intro. \\

\textcolor{blue}{\emph{
	We have tried to restructure the introduction through introducing a set of concrete questions towards the beginning of the manuscript that the model we introduce here can help to study. We have chosen not to focus on a single, published article in the introduction because we want to highlight the fact that the issues we raise with the dyadic framework are not just limited to one or two articles but are very much emblematic of how research into relational data structures is done across the field.
}} \\
