\section{Reviewer: 1}

By all rights, this should be a rejection but the technique introduced by the authors is interesting enough that I think it warrants a second chance. I have two problems with this manuscript. The first is that it seems to willfully ignore several large literatures that are closely related to the topic area. The second is that, though the basic framework of the statistical model is explained, non of its properties are derived and the statistics part of what is clearly a methods paper is curiously absent.

\subsection{Major Comments}

The first big problem with this manuscript is that there are vast swaths of literature that the authors ignore (almost entirely!) in order to claim to be the ``first'' to address this problem. They most certainly are not and need to beef up the front sections of the paper considerably in order to engage this literature and explain how their approach is different (which it is). Then, the backend of the paper needs to contrast their approach to the existing state of the art and show that their approach is better (which it very well may be, but one has no clue about this from reading the manuscript). \\

The first body of literature omitted from consideration is the growing literature on multiplex/multilayer networks. The second technique / approach / body of literature not engaged by this study is the stochastic actor oriented model by Snijders and Co. This is a dynamic model designed for the same purposes as the author's tensor model. This technique is widely used in the social sciences. The authors mention this in one sentence and then proceed as if it does not exist.

% >- Pretty much anything Peter Mucha has done since 2011
% >
% >- Multilayer networks M. Kivela, A. Arenas, M Barthelemy, J.P. Gleeson,
% >Y. Moreno and M. Porter, Journal of Complex Networks, Vol. 2, No. 3:
% >203-271 (2014)
% >
% >- Mathematical formulation of multi-layer networks M. De Domenico, A.
% >Sole-Ribalta, E. Cozzo, M. Kivela, Y. Moreno, M. A. Porter, S. Gomez and
% >A. Arenas, Physical Review X, 3, 041022 (2013)
% >
% >- Diffusion dynamics on multiplex networks S. Gomez, A. Diaz-Guilera, J.
% >Gomez-Gardenes, C.J. Perez-Vicente, Y. Moreno and A. Arenas, Physical
% >Review Letters, 110, 028701 (2013)

\textcolor{blue}{\emph{
	We agree that there are important pieces of the literature that we had not spent enough time discussing. We have added a substantially longer discussion to the second section of our paper that discusses the existing approaches and contrasts them to what the model we are introducing here provides. Specifically, we have added in a lengthier description of ERGM based approaches such as SAOM and TERGM, MRQAP (this was per the remarks of another reviewer), and the growing literature on multilayer networks. 
}} \\

The second major problem is that section 3 is highly dissatisfying. The authors explain the approach in very broad strokes but the technique remains mostly black-boxed. No properties of this technique are examined or proven. My intuition is that the technique probably works, but that hardly cuts muster in the statistical world. I would have thought maybe that there was a technical paper already published or in the works and that this is just an introduction/application paper, but no such
technical paper is referred to. Thus, without some sort of evidence that the technique works, as a responsible referee, I can only assume it does not. This must be rectified prior to this manuscript's consideration as a resubmission. \\

\textcolor{blue}{\emph{
	There is a technical paper forthcoming in the Annals of Applied Statistics that discusses the technical properties of the model we introduce here in much greater depth. This current JPR paper is meant to provide an introduction and application of this model for scholars working in international relations. 
}} \\

The application is kind of interesting, but, since it does not contrast the proposed technique to any of the other approaches to the same data/problem, one cannot really conclude anything from it. And one is left asking what it is doing there? \\

\textcolor{blue}{\emph{
	The extant approaches in the literature are not focused on running the type of model that we introduce here. There are a number of approaches available for studying longitudinal networks, the most popular of which include SAOM and TERGM, but to our knowledge none of these approaches provides an easy way for users to model the formation of endogenous, longitudinal networks -- the same is true for the multilayer/multiplex literature.  
	At the same time, as we note in the ``Performance analysis'' section of our paper that more work needs to be done in terms of predictive performance here. There are a number of avenues that we will be pursuing in future research to address this issue.
}} \\

