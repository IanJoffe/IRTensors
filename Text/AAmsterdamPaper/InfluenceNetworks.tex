%  MDW Version 0.0  details here
\documentclass[12pt,oneside,doublespace,pdflatex]{amsart}
%%% Bibliography
%\usepackage[sort&compress]{harvard}   
\usepackage{natbib}
\date{Amsterdam \today~version 0.01.}

%%% HEADERS & FOOTERS
\usepackage{fancyhdr}  
\pagestyle{fancy}  
\renewcommand{\headrulewidth}{0pt}  
%\lhead{ }\chead{Forecasting Conflict}\rhead{\today}
%\lfoot{\sc \tiny ISR}\cfoot{\thepage}\rfoot{\sc \tiny Submission}

%%% Tweaking margins, indents, FONTS, et cetera
\footskip=35pt
\parindent .49 in 
%\raggedright
\let\footnotesize\normalsize
\usepackage{geometry}
\geometry{verbose,tmargin=1in,bmargin=1in,lmargin=1in,rmargin=1in}
%\pdfpagewidth=8.5in % for pdflatex
%\pdfpageheight=11in % for pdflatex

%%% END Article customizations
%%% PACKAGES
\usepackage{booktabs} % for much better looking tables
\usepackage{array} % for better arrays (eg matrices) in maths
\usepackage{paralist} % very flexible & customisable lists (eg. enumerate/itemize, etc.)
\usepackage{verbatim} % adds environment for commenting out blocks of text & for better verbatim
%\usepackage{subfig} % make it possible to include more than one captioned figure/table in a single float
\usepackage[usenames]{color}
\usepackage{booktabs}
\usepackage{dcolumn}
\usepackage{tabularx,topcapt}
\newcolumntype{d}{D{.}{.}{-1}}
\usepackage{hyperref}
\usepackage{lscape}
\input xy
\xyoption {all}
\usepackage{wrapfig}
\usepackage{placeins}
\usepackage{color}
\usepackage{float}
\usepackage{amsfonts,amsmath,amssymb}
\usepackage{array, graphicx,psfrag,subfigure,tabularx,booktabs}
\usepackage{mparhack}
\usepackage{setspace}
\usepackage{natbib}
\usepackage{multicol}
\usepackage{endnotes}
\usepackage{dcolumn}
\usepackage{hyperref}
\usepackage{url}
\usepackage{stmaryrd}
\setcitestyle{authordate,round,semicolon,aysep={,},yysep={,}}
\bibpunct[:]{(}{)}{;}{a}{,}{,}
%\doublespacing
\graphicspath{{graphics/}}
\begin{document}

\title[Not Power]{Influence Networks in International Relations}

\author{Michael D. Ward}
%\thanks{}
\address{Michael D. Ward: Department of Political Science, Duke University, Durham, NC, USA, 27707\\}
\email{michael.d.ward@duke.edu}
 
\author{Peter D. Hoff}
%\thanks{}
\address{Peter D. Hoff: Department of Statistics, University of Washington, Seattle, WA, USA, 98195}
\email{pdhoff@u.washington.edu}

\author{Shahryar Minhas}
%\thanks{}
\address{Shahryar Minhas: Department of Political Science, Duke University, Durham, NC, USA, 27707\\}
\email{shahryar.minhas@duke.edu}

\begin{abstract}
There is a long history of power in international relations and world politics. As early as 275 BCE, the writings attributed to Chanakya (also known as Kautilya) were influential in discussing power between states.  In historical Chinese writing power is rarely discussed, but may be assumed. During the 16th century in Europe, Machiavelli famously wrote what a prince had to do to maintain his power. But conceptual definitions and quantitative measurements are much more recent, and emanate from a European tradition.  We show that there is no widely agreed upon definition of power that can be used in a principled, empirical fashion.  All extant, empirical measures focus either on a) material capabilities or b) mystical properties. We propose a relational, network model of influence relationships among international actors and derive its properties. We demonstrate its descriptive benefits as well as illustrate how it may be used to improve empirical studies which aspire to study power.
\end{abstract}
 
\pagestyle{fancy} \chead{} \rhead{}
\lhead{Influence in International Relations} \cfoot{\arabic{page}} \rfoot{}

\maketitle

\begin{quote} {\em
I was able to have a very successful career in political science without ever using the word ``power.'' You should do the same.}  Heinz Eulau,  $\sim$ 1984
\end{quote}

\section{Introduction}



\vskip .3cm
According to standard historical accounts,
the reign of Henry VIII of England (1509--1549) was largely devoted to preserving the {\em balance of power} by preventing Spain and France from joining forces and ruling Europe and conquering England.\footnote{Though maybe it was simply chronic traumatic encephalopathy.}  Thus, in many accounts of world politics beginning with the 16th Century and continuing until the end of empire, Britain was considered to be the ``balancer'' in world politics. And for many,  world politics
was thought to be a system of state led politics in which power was balanced \citep{spykman:1942,gulick:1955}.\footnote{Thucydides used this concept in his explanation for the onset of the
Peloponnesian Wars.  See also the essay by \cite{hume:1742}.} This, of course, was a very specific reading of history, one that is wildly at odds with contemporary historical analysis.\footnote{Henry VIII and Cardinal Wolsley are now thought
to have been more tactical than strategic, and not really interesting in balancing, but in protecting and in limited ways expanding England.  Even the Treaty of London was quickly abandoned and the Treaty of Bruges can be seen as pragmatic, not strategic. See \citet{raymond:2007} among others.  Most accounts of the balancing also fail to consider the role of the Holy Roman Empire as an actor in this play.}  However, the notion of the balance of power has survived to contemporary times and is well entrenched in the social sciences.\footnote{See \citet{zinnes:1967,dorussen:1999,fearon:1994,kadera:2001,moul:2005,nexon:2009,wolford:2015}, among a long list of others.} An early attempt to clarify the many different meanings of the concept of the balance is found in \citet{haas:1953}; however, the murkiness of this concept continues to this day.  A balance is a weighing, as in measurement, and requires some metric, however crude.  As yet, despite centuries of examining this question we have yet to arrive at an agreed to definition or metric of power.

Even apart from the concept of a balance, contemporary policy accounts often draw upon the concept of power, but rarely have a specific definition of what exactly this term refers to. Consider Angela Stint's recent analysis of what do do about Russia's military involvement in Syria 
\cite{stint:2016}. She notes
\begin{quote}
For all of Russia's domestic problems---a shrinking economy, a declining population, and high rates of capital flight and brain drain---it has projected a surprising amount of power not only in its neighborhood but also beyond. U.S. President Barack Obama may refer to Russia as a regional power, but Russia's military intervention in Syria demonstrates that it once again intends to be accepted as a global actor and play a part in every major international decision.
\end{quote}
But it is not clear exactly what is meant by Russia's power in this policy analysis by one of the leading scholars on Russian and Eastern European foreign policy. Is it the power to coerce Syria to ``stay the course'' against ISIS? Is the power to militarily defeat ISIS?  Is it the power to legitimize the Syrian government against it critics, internal and external? Is it simply the use of military force? In such policy analyses, power is often used to refer to influence, to victory in a conflict, to control over resources, and to status.  And, in many scholarly 
analyses power itself is a goal \citep{morgenthau:1948,waltz:1979}.  Often these meanings are woven together, somewhat selectively. 
\citet{gilpin:1975} has suggested that the way in which political scientists define and deal with the concept of power is
an ``embarrassment'' (page 24).  While pointing to many of these same issues, \citet{baldwin:1978} asserts
a widespread agreement on the necessity of using the concept of power to analyze international interactions. Both Gilpin and Baldwin suggest a relational approach to power as a compliment to the study of power as a material characteristic.
Yet, what this has mostly meant is direct comparison of the capability of one country with another, rather than an deeper consideration of the relational interpretation of power and strength.  
It has become a standard approach to use the ratio of capabilities for a measure of relative power in 
empirical studies 
\citep[, among others]{slantchev:2004,reed:etal:2008,butler:gates:2009,gartzke:weisiger:2014,carter:poast:2015}.

Looking back, the first quantitative measurement of power was stimulated by the work of \cite{fucks:1965} which was quickly followed up on in \cite{morgenstern:1974}.  Perhaps this lead to an empirical assessment of power that was largely based on capabilities, and material.  Certainly the availability of quantitative data on material characteristics of states was to influence
greatly the scholarship in this area and many scholars were to rely on the Correlates of War's Composite Index of National Capabilities (CINC) measures as a way of assessing power \citep{singer:etal:1972}.\footnote{Version 4.0 of these data, 
through 2007, are available at \url{http://correlatesofwar.org/data-sets/national-material-capabilities}. See also \citet{park:ward:1988}.}  This pushed scholarly consideration of power into a capabilities direction, rather than a direction in which power was seen as relational.  These approaches have implicitly assumed that power is material 
{\em and} fungible. If China has more capabilities than India, it has more power. If India and Japan have more capabilities than China, then they have more power. This kind of approach ignores the nuances of regional as well as global interactions, as well as ignoring the contexts in which states interact.  As such, it is easy to confound this understanding with simple examples. For example, since the US has more capabilities than North Korea, it has more power than North Korea and can prevent North Korea from doing something that it objects too, such as launching a satellite.\footnote{A kind of contrarian, network power was introduced in \cite{ward:house:1988,house:ward:1988} to capture the kind of power that rogue states may have.}

\section{Influence and Networks in International Relations}

A relational look a world affairs suggests, if not demands, a network perspective.  Fortunately in recent years there has been an increase in studies that examine the network of international relations.
Interestingly there was a brief period in the 1970s stimulated largely by
Johan Galtung's development of a structural theory of imperialism, which had crude networks, 
\citeyear{galtung:1971} in which there was considerable interests in networks as a way of understanding
the world system.  Early works include 
\cite{skjelsbaek:1972,chasedunn:rubinson:1977,bornschier:metal:1979,chirot:hall:1982}. These ideas largely grew into the
so-called world systems theory subfield of sociology, which thrives but is fairly isolated, even within sociology.

Fundamental work in physics energized the study of networks in the social world.  Following his dissertation on the topic,
Duncan Watts (with Steven Strogatz) introduced the idea of a small world \citep{watts:strogatz:1998} which captured widespread attention. This work was further generalized in \citet{newman:etal:2001} which permitted
a wide array of distributional assumptions.  The network phenomenon was popularized by 
\citet{watts:2004}, \cite{christakis:fowler:2009}, and the explosive popularity of Facebook and Twitter.  In a short decade,
networks moved from being an arcane, academic topic, to something that was becoming part of contemporary culture.

These developments in physics and sociology coupled with advances in led scholars to begin using
modern approaches to networks to study world affairs.  \citet{ward:hoff:etal:2003} used latent network
models to study interactions among nations, groups within nations, and individuals, focusing especially on Central Asia, including Afghanistan and surrounding regions. This was generalized and expanded in \citet{hoff:ward:2004} and applied to trade and conflict in \citep{ward:hoff:2007}. This latent approach--described in \citet{dorff:ward:2013}--was used to examine the Kantian piece \citep{ward:siverson:cao:2007}, policy networks \citep{cao:2009,cao:2010}, migration patterns \citep{breunig:cao:etal:2011}, international commerce \cite{ward:ahlquist:etal:2012}, rebellions \citep{metternich:dorff:etal:2013}, and human rights \citep{greenhill:2015}, among other substantive issues.

At the same time, network studies of world affairs also followed a longer standing tradition which was established in sociology, based largely on the idea that you could calculate sufficient statistics for networks, and include them in simple logistic regression models \citep{frank:1971,besag:1977b}.  This simple approach was overturned by modern statistical techniques, which permitted a general approach to studying what became known and random Markov fields \citep{besag:1985,besag:clifford:1989}. But for many, this rested on using network characteristics in estimating regression models. Among the first to bring this to international relations were Emily Hafner-Burton \citeyear{hafner-burton:2006,hafner-burton:kahler:etal:2009} and Zeev Maoz \citep{maoz:2006a,maoz:terris:etal:2007,maoz;2010}.  This approach became quite popular in political science broadly and in the study of world affairs, as well 
\citep[for example]{warren:2010,murdie:2014}

\citep{
lagazio:russett:2004,
cranmer:desmarais:2011,
ward:stovel:etal:2011,
cranmer:desmarais:etal:2012,
desmarais:cranmer:2012,
kinne:2012,
maoz2012special,
kinne:2013,
kinne:2014,
cranmer:desmarais:2015c,
cranmer:desmarais:2015c,
manger:pickup:2016,
chyzh:2017} 





\bibliographystyle{apsr}  %asa
\bibliography{/Users/mdw/git/whistle/master}
\newpage
\end{document}  \bye